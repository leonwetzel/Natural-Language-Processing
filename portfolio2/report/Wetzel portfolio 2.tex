% !TeX spellcheck = en_GB
\documentclass[a4paper, 11pt]{article}
\usepackage[english]{babel}
\usepackage{newtxtext,newtxmath}

\usepackage{amsmath}
\usepackage{mathtools}
\usepackage{dramatist}
\usepackage{dirtytalk}
\usepackage{multicol}
\usepackage{soul}
\usepackage{xcolor}

\usepackage[round]{natbib}

\usepackage{geometry}
\geometry{
	a4paper,
	total={150mm,257mm},
	top=20mm,
}

\usepackage{hyperref}
\hypersetup{
	colorlinks=true,
	linkcolor=blue,
	filecolor=blue,      
	urlcolor=blue,
	citecolor=blue
}

\setlength\parindent{0pt}

%opening
\title{\textbf{Natural Language Processing}\\Portfolio II}
\author{\textbf{Leon F.A. Wetzel}\\ Information Science \\ Faculty of Arts - University of Groningen\\ \texttt{l.f.a.wetzel@student.rug.nl}}

\begin{document}

\maketitle

\begin{abstract}
	
	In this document, you can find the results and explanations for the assignments of the second part of the portfolio for the course Natural Language Processing, taught at the University of Groningen. The corresponding Python code can be found at \url{https://github.com/leonwetzel/natural-language-processing}\footnote{All code will be published after the course has been completed}. Note that version control of Jupyter notebooks is done via \texttt{jupytext}, so do not forget to convert the relevant Python scripts to notebooks yourself!

\end{abstract}

\section{Week 1 - Regular Expressions}

asdadad

\end{document}
